\documentclass[oneside]{cv}

% \usepackage{biblatex}

% standard packages
\setlength\parindent{0in}
\usepackage{multirow}
\usepackage{multicol}
\usepackage{enumitem}
% Define command to insert month name and year as date
\usepackage{datetime}
\newdateformat{monthyear}{\monthname[\THEMONTH], \THEYEAR}

% Set fonts. Requires compilation with xelatex
\usepackage{fontspec}
\setmainfont{Helvetica Neue Light}[
  ItalicFont=Helvetica Neue Light Italic,
]

\newfontfamily\HelveticaNeueUltraLight{Helvetica Neue UltraLight}
\newfontfamily\HelveticaNeueUltraLightItalic{Helvetica Neue UltraLight Italic}


% shortcuts
\newcommand{\cv}{curriculum vitae}
\newcommand{\CV}{Curriculum Vitae}

% Personal Info
\newcommand{\firstname}{Lindsey}
\newcommand{\middleinitial}{J.}
\newcommand{\lastname}{Heagy}
\newcommand{\initials}{L.J.}

\newcommand{\emailaddress}{lindseyheagy@gmail.com}
\newcommand{\email}{\href{mailto:\emailaddress}{\emailaddress}}

\newcommand{\phone}{604-836-2715}

\newcommand{\websiteurl}{https://lindseyjh.ca}
\newcommand{\website}{\href{\websiteurl}{\websiteurl}}

\newcommand{\position}{PhD Candidate}
\newcommand{\affiliation}{University of British Columbia}

\newcommand{\fullname}{\firstname\ \middleinitial\  \lastname}
\newcommand{\shortname}{\initials\ \lastname}


% titles, headings, etc
\newcommand{\mytitle}[1]{
    {\HelveticaNeueUltraLight\fontsize{36pt}{0}\selectfont #1}\\[0.5cm]
}

\newcommand{\mysubtitle}{
    {\fontsize{14pt}{0}\selectfont \CV\ $~\cdot~$ \monthyear\today} \\[0.25cm]
}

\newcommand{\heading}[1]{
    \vspace{0.7cm}
    {\HelveticaNeueUltraLight\fontsize{18pt}{0}\selectfont #1}\\
    \vspace{-0.2cm}
    \hrule
    \vspace{0.4cm}
}

\newcommand{\subheading}[1]{
    \vspace{0.4cm}
    {\HelveticaNeueUltraLight\fontsize{14pt}{0}\selectfont #1}\\
    \vspace{-0.1cm}
}

\newcommand{\tworow}[1]{\multirow{2}{2cm}{#1}}

\usepackage{environ}
\NewEnviron{entryleft}{
    \vspace{-0.3cm}
    \begin{tabularx}{\textwidth}{@{}p{0.83\textwidth} p{0.17\textwidth}XX@{}}
        \BODY
    \end{tabularx}
    \vspace{-0.2cm}
}

\usepackage{environ}
\NewEnviron{entryright}{
    \vspace{-0.3cm}
    \begin{tabularx}{\textwidth}{@{}p{0.16\textwidth} @{}p{0.84\textwidth}}
        \BODY
    \end{tabularx}
    \vspace{-0.2cm}
}

\newenvironment{myquote}%
  {\list{}{\leftmargin=0.5cm\rightmargin=0cm}\item[]}%
  {\endlist}


\newcommand{\myindent}[1]{
    \begin{myquote}
    \vspace{-0.7cm}
        #1
    \vspace{-0.4cm}
    \end{myquote}
}

\NewEnviron{myitemize}{
    \vspace{-0.2cm}
    \begin{itemize}[topsep=0pt,itemsep=0.75pt]
        \BODY
    \end{itemize}
    \vspace{-0.2cm}
}

\NewEnviron{myenumerate}{
    % \vspace{-0.2cm}
    \begin{enumerate}[topsep=0pt, itemsep=0.75pt]
        \BODY
    \end{enumerate}
    % \vspace{-0.2cm}
}

% set header
\chead{
    \HelveticaNeueUltraLightItalic\fontsize{10pt}{12pt} \shortname $~\cdot~$
    \cv $~\cdot~$
    \monthyear\today
}

\begin{document}
\hyphenpenalty=100000
\exhyphenpenalty=1000000

% =========================== PERSONAL INFO ================================ %

\begin{centering}
    \mytitle{\fullname}
    \mysubtitle
    \position, \affiliation\\[0.1cm]
    \email $~\cdot~$ \phone $~\cdot~$ \website\\[0.5cm]
\end{centering}

% =========================== EDUCATION ================================ %

\heading{Education}

% \begin{tabularx}{\textwidth}{@{}p{0.9\textwidth} XX@{}R{0.1\textwidth}}
% \textbf{PhD} in Geophysics, University of British Columbia & \multirow{2}{2cm}{2012 -- \\ present}\\
% % $\quad$ University of British Columbia & \\
% $\quad$ Supervisor: Dr. Douglas Oldenburg & \\
% $\quad$ Thesis: \emph{Monitoring Hydraulic Fracturing with Electromagnetic Geophysics} & \\
% \end{tabularx}

\begin{entryright}
2012 -- present & \textbf{PhD} in Geophysics, University of British Columbia\\
& \myindent{Supervisor: Dr. Douglas Oldenburg}\\
& \myindent{Thesis: \emph{Monitoring Hydraulic Fracturing with Electromagnetic Geophysics}}
\end{entryright}

\begin{entryright}
 2008 -- 2012 & \textbf{BSc} with Honors in Geophysics, University of Alberta \\
& \myindent{First Class Honors}  \\
& \myindent{GPA: 4.0 / 4.0}
\end{entryright}


% =========================== EXPERIENCE ================================ %
\heading{Professional Experience}

\begin{entryright}
    \tworow{Apr. 2016 -- \\ Sept. 2017} & \textbf{Aranz Geo Canada Limited} (Calgary, AB) \\
    & Computational Geophysics Consultant (part-time)\\
    & \begin{myitemize}
        \item Consulting on software architecture and user interface design for Steno3D, a technical communication and 3D visualization software product
        \item Technical writing and editing, including internal reports and abstracts for scientific conferences
    \end{myitemize}
\end{entryright}


\begin{entryright}
    \tworow{Nov. 2015 -- \\ Apr. 2016} & \textbf{3point Science Inc} (Calgary, AB) \\
    & Computational Geophysicist (part-time) \\
    & \begin{myitemize}
        \item Consulting on the development and design of interactive 3D visualization software for the geosciences
    \end{myitemize}
\end{entryright}


\begin{entryright}
    \tworow{Jun. 2014 -- \\ Aug. 2014} & \textbf{Schlumberger Doll Research} (Boston, MA) \\
    & Geophysics Intern \\
    & \begin{myitemize}
        \item Supervisor: Dr. Dzevat Omeragic
        \item Examined upscaling techniques and performed numerical simulations to investigate the feasibility of electromagnetic imaging of complex hydraulic fractures
    \end{myitemize}
\end{entryright}


\begin{entryright}
    \tworow{Jun. 2013 -- \\ Aug. 2013} & \textbf{Schlumberger Electromagnetic Imaging} (Richmond, CA) \\
    & Geophysics Intern \\
    & \begin{myitemize}
        \item Supervisor: Dr. Michael Wilt
        \item Developed a workflow for mapping hydraulic fractures using cross-well electromagnetic surveys
        \item Awarded the patent: ``Determining proppant and fluid distribution'' (US Patent App. 14/494,313) which was developed from this work
    \end{myitemize}
\end{entryright}



\begin{entryright}
    \tworow{May 2012 -- \\ Aug. 2012} & \textbf{ConocoPhillips Canada} (Calgary, AB) \\
    & Geophysics Summer Student \\
    & \begin{myitemize}
        \item Supervisor: Richard Forest
        \item Interpreted 3D seismic volumes covering 8 townships in Western Canada by tying synthetic seismograms, mapping seismic horizons, and examining seismic attributes.
        \item Worked with geologists and reservoir engineers to map a potential natural gas resource and propose a drilling location
    \end{myitemize}
\end{entryright}



\begin{entryright}
    \tworow{May 2011 -- \\ Aug. 2011} & \textbf{Alfred Wegener Institute of Polar and Marine Research} (Bremerhaven, Germany) \\
    & Geophysics Summer Student \\
    & \begin{myitemize}
        \item Conducted numerical simulations to generate velocity profiles and estimate transport of the Antarctic Circumpolar Current south of Africa
        \item This project was funded through the Research Internships in Science and Engineering (RISE) program of the German Academic Exchange Service (DAAD)
        % \item Attended a program-wide meeting of interns from Canada, the United States, and the United Kingdom, discussing research programs and collaboration opportunities in Germany
    \end{myitemize}
\end{entryright}


% =========================== TEACHING ================================ %

\heading{Teaching Assistantships}

\begin{entryright}
2013 -- 2016 & \textbf{EOSC 350:  Environmental, Geotechnical, and Exploration Geophysics I} \\
& University of British Columbia \\
& \begin{myitemize}
    \item Instructor: Dr. Douglas Oldenburg
    \item Developed labs and assignments, including interactive numerical simulations for labs
    \item Coordinated content and website upgrades for the web-based resource ``Geophysics for Practicing Geoscientists'' (\href{http://gpg.geosci.xyz}{http://gpg.geosci.xyz})
    \item Worked with a Teaching Assistant Team of 5-6 members to instruct labs, mark assignments \& exams for 50-60 geology and engineering students
\end{myitemize}
\end{entryright}

\begin{entryright}
2015 & \textbf{Directed Studies: Inversion in Applied Geophysics} \\
& University of British Columbia \\
& \begin{myitemize}
    \item Instructor: Dr. Douglas Oldenburg
    \item Provided guidance for an undergraduate student in his independent study. He developed Jupyter Notebook Tutorials on the basics of geophysical inversions (\href{https://github.com/jokulhaup/directed_studies}{https://github.com/jokulhaup/directed\_studies})
\end{myitemize}
\end{entryright}


\begin{entryright}
2012 & \textbf{EOSC 354: Analysis of Time Series and Inverse Theory for Earth Scientists} \\
& University of British Columbia \\
& \begin{myitemize}
    \item Instructor: Dr. Michael Bostock
    \item Instructed labs, marked labs and assignments for a class of 14 geophysics students
\end{myitemize}
\end{entryright}


% =========================== Service ================================ %
\heading{Service and Outreach}

\begin{entryright}
2017 & \textbf{Journal of Open Source Software} \\
& Editor: Geoscience, geophysics (\href{http://joss.theoj.org/about}{http://joss.theoj.org/about})
\end{entryright}

% \begin{tabular}{p{13cm} R{2.75cm}}
% \textbf{Syzygy Data \& Computing, a CFI Cyber infrastructure Proposal} & 2017 \\[0.1cm]
% Canada Foundation for Innovation (CFI) Grant Application &\\
% \end{tabular}
% \begin{tabular}{p{0.05cm} p{13.5cm}}
% & Proposal Committee Member \\
% & Principal Investigator: Dr. Michael Lamoureux, University of Calgary \\
% & Syzygy: \href{http://syzygy.ca/}{http://syzygy.ca/}, CFI: \href{https://www.innovation.ca/}{https://www.innovation.ca/}
% \end{tabular}

% \begin{itemize}
%     \item Principal Investigator: Dr. Michael Lamoureux, University of Calgary
%     % % \vspace{-6px}
%     \item Proposal Committee Member
%     % % \vspace{-6px}
%     \item Syzygy: \href{http://syzygy.ca/}{http://syzygy.ca/}, CFI: \href{https://www.innovation.ca/}{https://www.innovation.ca/}\\
% \end{itemize}

% % \vspace{7px}

\begin{entryright}
    2017 & \textbf{Society of Exploration Geophysics Distinguished Instructor Short Course 2017} \\
    & Support Instructor for \emph{Geophysical Electromagnetics: Fundamentals and Applications} by Dr. Douglas Oldenburg (\href{http://disc2017.geosci.xyz}{http://disc2017.geosci.xyz}) \\
    & \begin{myitemize}
        \item{Included generating course material, leading tutorials on numerical simulations and inversions for electromagnetics, addressing questions from participants, documenting the events and participant presentations on a blog (\href{https://medium.com/disc2017}{https://medium.com/disc2017}) and developing and maintaining the course website (\href{https://disc2017.geosci.xyz}{https://disc2017.geosci.xyz})}
        \item{Locations for which I was a support instructor:
            \begin{myitemize}
                \vspace{0.05cm}
                \item Denver, USA (January 30-31, 2017)
                \item Perth, Australia (July 27-28, 2017)
                \item Adelaide, Australia (August 2-3, 2017)
                \item Brisbane, Austraila (August 7-8, 2017)
                \item Delft, Netherlands (September 11-12, 2017)
                \item Bonn, Germany (September 18-19, 2017)
                \item Vienna, Austria (September 21-22, 2017)
                \item Zurich, Switzerland (September 26-27, 2017)
                \item Aarhus, Denmark (October 2-3, 2017)
                \item Toronto, Canada (October 27, 2017)
                \item Mexico City, Mexico (November 6-7, 2017)
                \item Buenos Aires, Argentina (November 13-14, 2017)
                \item Santiago, Chile (November 16-17, 2017)
                \item Santa Cruz de la Sierra, Bolivia (November 22-23, 2017) - Cancelled
                \item Rio de Janeiro, Brazil (November 28-29, 2017)
                \item Calgary, Canada (December 5-6, 2017)
                \item Vancouver, Canada (December 12-13, 2017)
            \end{myitemize}
        }
    \end{myitemize}
\end{entryright}

\begin{entryright}
    2017 & \textbf{JupyterCon}, August 22-25, New York, NY \\
    & Program Committee Member (\href{https://conferences.oreilly.com/jupyter/jup-ny}{https://conferences.oreilly.com/jupyter/jup-ny}) \\
    & \begin{myitemize}
        \item{This was the inaugural year for JupyterCon}
        \item{Committee responsibilities included: reviewing abstracts, outreach to potential speakers, event promotion}
    \end{myitemize}
\end{entryright}

\begin{entryright}
    2016 & \textbf{Banff International Research Station: Geophysical Simulation and Inversion Workshop}, August 19-21, Banff, AB  \\
    & Supporting Organizer with Dr. Douglas Oldenburg, Dr. Adam Pidlisecky and Rowan Cockett (\href{http://www.birs.ca/events/2016/2-day-workshops/16w2695}{http://www.birs.ca/events/2016/2-day-workshops/16w2695})
\end{entryright}

\begin{entryright}
2014 -- present & \textbf{GeoSci.xyz} \\
& Core maintainer and contributor to online interactive textbooks for geophysics (\href{http://geosci.xyz}{http://geosci.xyz}). Resources include: \\
& \begin{myitemize}
    \item Geophysics for Practicing Geoscientists (\href{http://gpg.geosci.xyz}{http://gpg.geosci.xyz})
    \item Electromagnetic Geophysics (\href{http://gpg.geosci.xyz}{http://gpg.geosci.xyz})
    \item Computational Geophysics (\href{http://computation.geosci.xyz}{http://computation.geosci.xyz})
\end{myitemize}
\end{entryright}

\begin{entryright}
2014 -- present & \textbf{SimPEG} \\
& Core maintainer and contributor to the open source software (\href{https://github.com/simpeg}{https://github.com/simpeg}) \\
& Community developer including leading weekly meetings and fielding user questions (google group: \href{https://groups.google.com/forum/\#!forum/simpeg}{https://groups.google.com/forum/\#!forum/simpeg}, slack: \href{http://slack.simpeg.xyz}{http://slack.simpeg.xyz})
\end{entryright}

\begin{entryright}
2014 -- 2015 & \textbf{Undergraduate Research Mentor} \\
& Advised Mohamed Rassas on as a part of the Undergraduate Research Opportunities Research Experience Program at the University of British Columbia\\
& \begin{myitemize}
    \item His work resulted in the presentation \emph{A comparison of conventional and open channel hydraulic fracturing and the importance of imaging to optimize the fracturing process} at the Multidisciplinary Undergraduate Research Conference at the University of British Columbia
\end{myitemize}
\end{entryright}

\begin{entryright}
2005 -- 2009 & \textbf{Volunteer Instructor: Alberta Diploma Exam Reviews} \\
& Developed and delivered review courses for Physics 30, Chemistry 30, Pure Math 30\\
& Supervised by Mr. David Westra
\end{entryright}






% =========================== CONTRIBUTIONS ================================ %

\heading{Publications}

% \bibliography{./refs/lindseyrefs}
\nobibliography{./refs/lindseyrefs}
\bibliographystyle{seg}

\subheading{Peer Reviewed Publications}

    \begin{myenumerate}
        \item \bibentry{heagy2017electromagnetics}
        \item \bibentry{CaudilloMata2016}
        \item \bibentry{cockett2015simpeg}
    \end{myenumerate}

% \subheading{Peer Reviewed Publications (submitted or in review)}

%     \begin{myenumerate}
%         \item \bibentry{Heagy2016}
%     \end{myenumerate}

\subheading{Non Peer Reviewed Publications}

    \begin{myenumerate}
        \item \bibentry{kang2017}
        \item \bibentry{Cockett2016}
    \end{myenumerate}

\subheading{Patents}

    \begin{myenumerate}
        \item \bibentry{wilt2014determining}
    \end{myenumerate}

% \subheading{Non-Peer Reviewed Publications}



\subheading{Conference Proceedings}

    \vspace{-0.2cm}
    ~($*$ : invited,  $\dagger$ : award)
    \vspace{0.3cm}

    \begin{myenumerate}
        \item \bibentry{heagy2017}
        \item $*$ \bibentry{kang2016}
        \item $*$ \bibentry{heagy2016a}
        \item $*$ \bibentry{heagy2016b}
        \item \bibentry{yang20163d}
        \item \bibentry{heagy2015simpegem}
        \item \bibentry{heagy2015real}
        \item \bibentry{cockett2015development}
        \item \bibentry{heagy2015modelling}
        \item \bibentry{kang2015moving}
        \item \bibentry{cockett2014simpeg}
        \item \bibentry{heagy2014parametrized}
        \item \bibentry{caudillo2014numerical}
        \item \bibentry{fournier2014multi}
        \item \bibentry{devriese2014magnetic}
        \item \bibentry{wilt2014hydrofracture}
        \item{$\dagger$ \bibentry{heagy2014does} \\ $~\quad ^\dagger$ Student Honourable Mention Integrated Poster}
        \item{$\dagger$ \bibentry{heagy2013investigating} \\  $~\quad ^\dagger$ Award of Merit (Best Student Paper, Annual Meeting)}
    \end{myenumerate}

\subheading{Other Conference Presentations}

    \begin{myenumerate}
        \item \bibentry{heagy2017jupytercon}
        \item \bibentry{heagy2017bcgs}
        \item \bibentry{heagy2016scipy}
        \item \bibentry{heagy2015scipy}
        \item \bibentry{rosenkjaer2015scipy}
        \item{$*$ \bibentry{heagy2014dnp} \\ $~\quad ^*$ Invited to ``Best of the Development and Production Forum'' at the SEG 2014 Annual Meeting}
    \end{myenumerate}


\heading{Software Contributions}

\begin{entryright}
2014 -- present & \textbf{SimPEG} \\
& \myindent{Software for numerical simulations and inversions in geophysics} \\
& \myindent{\href{https://github.com/simpeg/simpeg}{https://github.com/simpeg/simpeg}}
\end{entryright}

\begin{entryright}
2014 -- present & \textbf{discretize} \\
& \myindent{Discretization tools for finite volume and inverse problems} \\
& \myindent{\href{https://github.com/simpeg/discretize}{https://github.com/simpeg/discretize}}
\end{entryright}

\begin{entryright}
2016 -- present & \textbf{geoana} \\
& \myindent{Analytic solutions in geophysics}\\
& \myindent{\href{https://github.com/simpeg/geoana}{https://github.com/simpeg/geoana}}
\end{entryright}

% =========================== AWARDS ================================ %

\heading{Awards and Scholarships}

\begin{entryright}
2016 & \textbf{UBC Library: Innovative Dissemination of Research Award}  \\
& \myindent{Awarded for the SimPEG framework and community development ($\$1,000$). With Rowan Cockett and Seogi Kang.}
\end{entryright}

\begin{entryright}
2014 -- 2017 & \textbf{NSERC Vanier Scholarship} \\
& \myindent{Vanier Scholars demonstrate leadership skills and a high standard of scholarly achievement in graduate studies in the social sciences and/or humanities, natural sciences and/or engineering and health. The Vanier Scholarship is the top graduate scholarship in Canada. ($ \$50,000 \times 3$)}
\end{entryright}

\begin{entryright}
2014 -- 2017 & \textbf{Alexander Graham Bell Canada Graduate Scholarship} \\
& \myindent{Awarded to high caliber scholars who are engaged in a doctoral program in the natural sciences or engineering (Declined) ($ \$35,000 \times 3$)}
\end{entryright}

\begin{entryright}
2014 -- 2018 & \textbf{Four Year Fellowship (FYF) for PhD Students} \\
& \myindent{Selection based on academic excellence, upon the recommendation of the graduate program at UBC ($\$18,000 \times 4$, declined 3/4)}
\end{entryright}

\begin{entryright}
2013 & \textbf{Special UBC Graduate Scholarship - W.H. Mathews Scholarship} \\
& \myindent{Awarded for academic achievement in Earth, Ocean and Atmospheric Sciences at UBC ($\$5,000$)}
\end{entryright}

\begin{entryright}
2012 & \textbf{Governor General’s Silver Medal} \\
& \myindent{Awarded annually to the three undergraduate students (institution-wide) who achieve the highest academic standing overall upon graduation from his/her Bachelor degree program (University of Alberta)}
\end{entryright}

\begin{entryright}
2012 & \textbf{Lieutenant-Governor’s Gold Medal} \\
& \myindent{Awarded to the convocating student from an Honours program in the Faculty of Science who has shown the highest distinction in scholarship (University of Alberta)}
\end{entryright}

\begin{entryright}
2012 & \textbf{APEGGA Past Presidents’ Medal in Geophysics} \\
& \myindent{Awarded to the convocating student who is a Canadian Citizen or Permanent Resident with the highest academic standing in a specialization or honours program in Geophysics on the basis of the final year}
\end{entryright}

\begin{entryright}
2011 & \textbf{The APEGGA Scholarship in Geophysics} \\
& \myindent{Awarded on the basis of superior academic achievement in Honors Geophysics or Specialization in Geophysics ($\$3,000 \times 2$)}
\end{entryright}

\begin{entryright}
2010 -- 2012 & \textbf{The David K Robertson Award in Geophysics and Geology} \\
& \myindent{Awarded to a student entering the third year of a BSc Specializing in Geology or Geophysics on the basis of passion and talent in their field of study, demonstrated leadership, participation in extracurricular activities, and academic standing. ($\$5,000 \times 2$)}
\end{entryright}

\begin{entryright}
2010 -- 2012 & \textbf{The Encana Geology and Geophysics Scholarship} \\
& \myindent{Awarded to student(s) with superior academic achievement entering the third or fourth year of study for a Bachelor of Science with a major in Geology or Geophysical Sciences. ($\$3,500 \times 2$)}
\end{entryright}

\begin{entryright}
2009 -- 2011 & \textbf{Louise McKinney Post Secondary Scholarship, Government of Alberta} \\
& \myindent{Recognizes students for their academic achievements at a provincial level and encourages them to continue in their undergraduate program of study ($\$2,500 \times 3$)}
\end{entryright}

\begin{entryright}
2009 & \textbf{Pearl Cuthbertson Memorial Award} \\
& \myindent{Awarded to a student entering the second year of study for a Bachelor of Science degree who has completed Science 100. Selection based on academic standing and demonstrated determination, curiosity and enthusiasm for science. ($\$2,000 \times 2$)}
\end{entryright}

\begin{entryright}
2009 & \textbf{Pearson Book Prize}\\
& \myindent{Awarded for academic achievement in Writing Studies in Science 100}
\end{entryright}

\begin{entryright}
2008 -- 2012 & \textbf{Dean's Honor Roll, University of Alberta} \\
& \myindent{Awarded for academic achievement ($\times 4$)}
\end{entryright}


% =========================== Grants ================================ %

\heading{Grants}

\begin{entryright}
2014 & \textbf{Science Center for Learning and Teaching - Development Grant} \\
& \myindent{Development of online interactive resources for undergraduate geophysics at UBC ($\$2,500$)} \\
& \myindent{Principal Investigator: Dr. Douglas Oldenburg} \\
\end{entryright}


% =========================== Grants ================================ %

\heading{Media}

\begin{entryright}
Apr. 24, 2017 & Guest on Episode 41, \emph{Undersampled Radio} by Graham Ganssle and Matt Hall (\href{https://undersampledrad.io/home/2017/4/inverterizer}{https://undersampledrad.io})
\end{entryright}

\begin{entryright}
Jan. 24, 2017 & Guest on Episode 11, \emph{Seismic Soundoff} by the Society of Exploration Geophysicists (\href{http://seg.org/podcast/Post/4610/Episode-11-Geophysical-Electromagnetics-2017-DISC}{http://seg.org/podcast})
\end{entryright}

\begin{entryright}
Jun. 7, 2012 & Article: \emph{Science 100 pioneer grounded in geophysics - University of Alberta Spring Convocation 2012: Celebrating Talented People} (\href{https://www.ualberta.ca/news-and-events/newsarticles/2012/06/science100pioneergroundedingeophysics}{https://www.ualberta.ca/news-and-events/newsarticles})
\end{entryright}

% \begin{tabular}{p{2.5cm} p{12cm}}
%     Apr. 24, 2017 & Guest on Episode 41, \emph{Undersampled Radio} by Graham Ganssle and Matt Hall (\href{https://undersampledrad.io/home/2017/4/inverterizer}{https://undersampledrad.io})\\[0.2cm]
%     Jan. 24, 2017 & Guest on Episode 11, \emph{Seismic Soundoff} by the Society of Exploration Geophysicists (\href{http://seg.org/podcast/Post/4610/Episode-11-Geophysical-Electromagnetics-2017-DISC}{http://seg.org/podcast})\\[0.2cm]
%     Jun. 7, 2012 & Article: \emph{Science 100 pioneer grounded in geophysics - University of Alberta Spring Convocation 2012: Celebrating Talented People} (\href{https://www.ualberta.ca/news-and-events/newsarticles/2012/06/science100pioneergroundedingeophysics}{https://www.ualberta.ca/news-and-events/newsarticles}) \\[0.2cm]
% \end{tabular} \\[0.2cm]

% \subsection{Outreach and Other Contributions}

    % \begin{enumerate}
    %     \item{ 2015--}
    %     \item \bibentry{emgeosci}
    %     % \item \bibentry{gpg}
    %     % \item \bibentry{simpegsoftware}
    % \end{enumerate}


\heading{Professional Development}


\subheading{Conferences Attended}

\begin{entryright}
    2017         & EM-6: The 6th International Symposium in Three-Dimensional Electromagnetics
\end{entryright}

\begin{entryright}
    2016         & FORCE11: The Future of Research Communications and e-Scholarship Meeting
\end{entryright}

\begin{entryright}
    2015 -- 2016 & (2) SciPy Conference
\end{entryright}

\begin{entryright}
    2014 -- 2015 & (2) British Columbia Geophysical Society EM Workshop
\end{entryright}

\begin{entryright}
    2014         & SEG Development and Production Forum
\end{entryright}

\begin{entryright}
    2014         & GeoConvention
\end{entryright}

\begin{entryright}
    2014 -- 2016 & (3) AGU Annual General Meeting
\end{entryright}

\begin{entryright}
    2011 -- 2016 & (6) Society of Exploration Geophysics Annual Meeting
\end{entryright}



\subheading{Courses Attended}

\begin{entryright}
    2015 & \emph{Presenting Data and Information} by Edward Tufte
\end{entryright}

\begin{entryright}
    2014 & SEG Distinguished Instructor Short Course: \emph{Microseismic Imaging of Hydraulic Fracturing: Improved Engineering of Unconventional Shale Reservoirs} by Shawn Maxwell
\end{entryright}

% \nobibliography{lindseyrefs}
\bibliographystyle{seg}
% \bibliographystyle{}
\end{document}
